%% ----------------------------------------------------------------
%% main.tex -- MAIN FILE (the one that you compile with LaTeX)
%% ----------------------------------------------------------------
%
%
% LaTeX main file example for UTEC thesis format.
% Created by Victor Murray in collaboration with Oscar Ramos and Juan Carlos Barbaran.
% June, 2021
%
%

\documentclass[a4paper, 12pt, oneside]{tesisutec}

% Color commands
\usepackage{xcolor}
\newcommand{\leonidas}[1]{\textcolor{orange}{#1}}
\newcommand{\jorge}[1]{\textcolor{magenta}{#1}}


% Location of the graphics files (set up for graphics to be in PDF format)
\graphicspath{images/}

% Include any extra LaTeX packages required
\usepackage[comma, sort&compress]{natbib}
\usepackage{verbatim}  %
\usepackage[spanish]{babel}
\selectlanguage{spanish}

\makeatletter
% Reinsert missing \algbackskip
\def\algbackskip{\hskip-\ALG@thistlm}
\makeatother

\usepackage{float}
\usepackage[english,onelanguage,ruled,vlined, linesnumbered]{algorithm2e}

\hypersetup{urlcolor=blue, colorlinks=true}


\usepackage{apalike}

\usepackage{notoccite}  % Librería necesaria para arreglar el orden de referencias en overleaf.com
\usepackage{hyperref} % Librería para hacer 'clickable' todas las referencias. OPCIONAL.

% For Spanish
%\usepackage[utf8]{inputenc}
% For correct copy of tildes in PDF
\usepackage[T1]{fontenc}

%% ============================================================================
\begin{document}

\frontmatter
\department{CIENCIA DE LA COMPUTACIÓN}
\degree{Licenciado en Ciencia de la Computación}

\title{Diseño de dispositivos nanofotónicos resilientes a errores de fabricación usando algoritmos de optimización y computación de alto desempeño}
%\title{Estudio de Cinco Algoritmos de Optimización usando HPC para Fabricar Dos Dispositivos Nanofotónicos: \emph{bend-90°} y \emph{WDM}}
\author{José Leonidas García Gonzales}
\authorid{201720102}
\supervisor{Jorge Luis Gonzalez Reaño}
\date{\today}

\maketitle
\setstretch{1.5}

% TODO: DESCOMENTAR PARA PFC3
%\input{header/approvement.tex}
%\input{header/dedicatory.tex}
%%% ============================================================================
%  Página de Agradecimientos
%% ============================================================================

\acknowledgements{
  En primer lugar gracias a mi familia por todo el apoyo brindado.
  Este trabajo no se hubiera podido completar sin su infinita paciencia, 
  constante soporte y deliciosas comidas.

  Mi más profundo agradecimiento al Dr. Jorge Gonzalez por todo el esfuerzo
  brindado en la realización de este trabajo. Asimismo, gracias a la Dr. Ruth Rubio
  y al Ing. Roy Prosopio por el constante apoyo con su conocimiento en esta área de investigación.

  Finalmente, gracias a todos los profesores que se tomaron la molestia de leer mi trabajo,
  las recomendaciones brindadas fueron de gran utilidad.
}



\tableofcontents
\newpage
\listoftables
\newpage
\listoffigures

\addtocontents{toc}{\vspace{1.5em}}

%% ============================================================================
\mainmatter
\pagestyle{fancy}

%\chapter*{\center \Large \vspace{-4.5cm} RESUMEN}
\addcontentsline{toc}{section}{\bfseries RESUMEN}
\markboth{RESUMEN}{RESUMEN} 

La fotónica en silicio es un área en desarrollo emergente y constante en las últimas décadas. Los dispositivos fotónicos muestran potencial de aplicación para mejorar
los sistemas de cómputo, telecomunicaciones y otras áreas.
Sin embargo, aún es un reto integrar una gran cantidad de dispositivos fotónicos fundamentales en un chip con área reducida y baja pérdida. 
En el presente trabajo se diseñaron dos dispositivos fundamentales: (i) \emph{bend} y (ii) \emph{2-channel wavelength-demultiplexer} (WDM).
Los diseños se realizaron en un área de $2 \mu m \times 2 \mu m$
siguiendo una estrategia basada en optimización topológica robusta.

Realizamos la evaluación y comparativa de cinco algoritmos de optimización de primer orden: 
(i) \emph{Limited-memory Broyden–Fletcher–Goldfarb–Shanno with boundaries} (L-BFGS-B), 
(ii) \emph{Method of Moving asymptotes} (MMA), 
(iii) \emph{Covariance Matrix Adapatation Evolution Strategy} (G-CMA-ES), (iv) \emph{Gradient Particle Swarm Optimization} (G-PSO) y (v) \emph{Gradient Genetic Algorithm} (G-GA). Los últimos tres algoritmos son variaciones propuestas a sus versiones
más populares (CMA-ES, PSO y GA) donde se incluye el cálculo de
la gradiente para guiar su proceso de optimización.

En nuestros resultados los diseños mejor optimizados presentan: 
(i) transmitancias mayores al $90 \%$ y robustez ante errores de fabricación de dilatación y erosión, 
(ii) porcentaje de gris menor al $2 \%$ y 
(iii) desempeño consistente y con cambios suaves en un rango de longitudes de onda (1500-1600 nm \emph{bend} y 1250-1600 nm WDM) incluso si se eliminan sus regiones no conexas. 

Estos resultados son prometedores para 
(i) la integración de dispositivos  WDM en un área menor al reportado en el estado del arte ( < $2.8 \mu m \times 2.8 \mu m$) y 
(ii) el diseño de \emph{bends} con menores pérdidas que el diseño intuitivo-tradicional de $1 \mu m$ de radio.

\noindent \textbf{Palabras clave:}\\
\noindent Algoritmos de Optimización, Diseño Inverso, Fotónica Integrada, Métodos Numéricos, Optimización Topológica Robusta.

%\chapter*{\center \Large \vspace{-5.5cm} ABSTRACT}
\addcontentsline{toc}{section}{\bfseries ABSTRACT}
\markboth{ABSTRACT}{ABSTRACT} 

\begin{center}
\Large \vspace{-1.5cm} \textbf{Evaluation of First-Order Optimization Algorithms in Robust Topology Optimization of Nanophotonic Devices}
\end{center}

Silicon photonics is an emerging area with constant growth in the last decades.
Photonic devices show potential applications to improve computing systems, telecommunications and other areas.
Nevertheless, it is still a challenge to integrate a great number of fundamental photonic
devices in a chip with small area and low loss.
In this work we designed two fundamental photonic devices: (i) bend and (ii)
2-channel wavelength-demultiplexer (WDM).
The designs were done on a $2 \mu m \times 2 \mu m$ area following a robust topology
optimization based strategy.

We evaluated and comparated five first-order optimization algorithms:
(i) \emph{Limited-memory Broyden–Fletcher–Goldfarb–Shanno with boundaries} (L-BFGS-B), 
(ii) \emph{Method of Moving asymptotes} (MMA), 
(iii) \emph{Covariance Matrix Adapatation Evolution Strategy} (G-CMA-ES), (iv) \emph{Gradient Particle Swarm
Optimization} (G-PSO) and (v) \emph{Gradient Genetic Algorithm} (G-GA). 
The last three algorithms are variations of their more standard versions (CMA-ES, PSO and GA)
where the computation of the gradient is included to guide the optimization process.

The best optimized designs show:
(i) transmission greater than $90 \%$ and robustness to over/under-etching,
(ii) a gray percentage of less than $2 \%$ and
(iii) their performance is broadband consistent with smooth changes 
(1500-1600 nm \emph{bend} and 1250-1600 nm WDM) even after deleting non-convex regions.

These results are promising for 
(i) the integration of WDM devices in an lower area than 
state of the art (< $2.8 \mu m \times 2.8 \mu m$) and 
(ii) the design of bends with lower loss than intuitive-traditional designs of $1 \mu m$ radius.

\noindent \textbf{Keywords:}\\
\noindent Optimization Algorithms; Inverse Design; Integrated Photonic; Numeric Methods; Robust Topology
Optimization.


\chapter{Motivación y Contexto}

Durante los últimos años distintas áreas de la industria han mostrado interés en la fotónica por el elevado ancho de banda que ofrece en comunicaciones.
Sin darnos cuenta, esta tecnología se ha ido integrando a nuestro día a día comenzando con el éxito de los cables ópticos.
Probablemente, la creciente popularidad de esta área es debido a que el siguiente paso lógico para mejorar el rendimiento de los dispositivos electrónicos, los cuales están llegando a sus límites físicos, parece ser la incorporación o acondicionamiento de estos con dispositivos fotónicos \citep{Glick2018, LukasChrostowski2010}.
Sin embargo, la convergencia de estas tecnologías aún presenta muchos desafíos. Particularmente, existen dispositivos como el \emph{bend-90°} y \emph{2-splitter} que aún no logran aplicación industrial aún cuando son fundamentales para una variedad de circuitos fotónicos \citep{Molesky2018}.


La dificultad de encontrar aplicación industrial en parte se debe que al trabajar en la escala de nanómetros los diseños intuitivos han mostrado un bajo rendimiento y poca flexibilidad para incorporar restricciones de fabricación. 
Debido a ello se comenzó a aplicar el diseño inverso. 
Con esta metodología primero se define las propiedades deseadas de un dispositivo y luego se busca una geometría que las satisfaga.
De esta manera se ha logrado obtener diseños no intuitivos pero con un buen rendimiento \citep{Su2020}.

El diseño inverso ha recibido mucha atención en fotónica durante los últimos 20 años \citep{Molesky2018}. 
Trabajos como los de \cite{Su2020} han logrado encontrar diseños de un \emph{bend-90°} y \emph{2-splitter} con un buen rendimiento de acuerdo a simulaciones.
Luego, investigaciones como las de \cite{Su2018} y \cite{Piggott2017} intentan incorporar restricciones de fabricación para obtener dispositivos que al fabricarse mantengan un buen rendimiento. 
No obstante, la búsqueda de estos diseños suele realizarse aplicando algoritmos cuya selección y configuración es mayormente debido a un tedioso proceso de prueba y error.
Así, han surgido estudios como los de \cite{Schneider2019, Elsawy2020, Gregory2015} quienes buscan comparar distintos algoritmos de optimización en el diseño inverso de dispositivos fotónicos.


Aún cuando una cantidad considerable de investigaciones están usando el diseño inverso para optimizar dispositivos fotónicos, existe una carencia de estudios de comparación de algoritmos de optimización aplicados a un \emph{bend-90°} y \emph{2-splitter}. 
Es importante la realización de ellos pues cada dispositivo es una clase distinta de problema \citep{Molesky2018}. 
Así, el objetivo principal de este estudio es cubrir esta brecha evaluando el rendimiento y convergencia de algoritmos de optimización usados para estos dispositivos.

El presente trabajo está organizado de la siguiente manera:

El capítulo 1 brinda una introducción al tema de investigación, describe el problema a detalle, justifica la relevancia de resolver el problema, define los objetivos y señala los aportes del trabajo.

El capitulo 2 desarrolla conceptos y terminología necesaria para entender las siguientes capítulos.

El capítulo 3 describe trabajos relacionados del estado del arte.

\section{Introducción}

La fotónica está atrayendo el interés de la industria debido a su potencial en términos de escalabilidad y los beneficios de costo-eficiencia. 
Este potencial es evidente, por ejemplo, con los siguientes tres puntos. 
Primero, si se quiere mantener la tendencia que cada 10 años se mejore en un factor de 1000 el rendimiento de los sistemas electrónicos, entonces parece ser indispensable la convergencia de estos con sistemas fotónicos \citep{Glick2018}. 
Segundo, existe una inversión billonaria en la fabricación de transistores cuyos procesos se están comenzando a lograr adaptar para fabricar circuitos fotónicos \citep{LukasChrostowski2010}.
Tercero, el elevado ancho de banda que ofrece en comunicaciones digitales ha despertado el interés de cinco grandes sectores de producción: i) centrales de datos, ii) intenet de las cosas, iii) sector automotriz, iv) sensores, v) fotónica RF \citep{LukasChrostowski2010, Glick2018}.

Los dispositivos fotónicos se utilizan en grandes cantidades en los circuitos fotónicos integrados \citep{LukasChrostowski2010}. 
Estos trabajan en la escala de nanómetros y son diseñados para funcionar bajo ciertas especificaciones. 
Así, para cumplir los requerimientos deseados existen dos estrategias comunes: diseño tradicional y diseño inverso \citep{Molesky2018}.


En el diseño tradicional se define el dispositivo con geometrías simples que permiten obtener funciones analíticas de sus propiedades físicas. 
Esto se realiza para poder optimizar la función obtenida a partir de los parámetros que la definan. 
Dicha optimización se suele ejecutar haciendo un barrido de los parámetros, con algoritmos genéticos o usando \emph{particle swarm optimization}. 
Es un enfoque simple, pero que ha obtenido buenos resultados. 
Sin embargo, existen tres grandes inconvenientes con este planteamiento. 
Primero, solo estamos explorando una pequeña fracción de todos los posibles diseños.
Segundo, por lo general no es conocido el límite de rendimiento del dispositivo.
Tercero, al trabajar en la escala de nanómetros, existen casos como el \emph{bend-90°} y \emph{2-splitter} que han presentado un bajo rendimiento con diseños tradicionales \citep{Molesky2018, Su2020}.


En el diseño inverso se busca hacer una mayor exploración de todos los posibles diseños. 
Para ello, ya no nos limitamos a solo usar diseños intuitivos, ver figura \ref{fig:devices}. Ahora, definimos geometrías arbitrarias y usamos simulaciones computacionales para determinar las propiedades físicas del dispositivo \citep{Molesky2018, Su2020}. Este enfoque ha logrado conseguir mejores resultados que los obtenidos por el diseño tradicional \citep{Su2018, Molesky2018}. Sin embargo, este planteamiento viene acompañado de nuevos desafíos.


\begin{figure}[h]
  \centering
  % 1° row
  % Traditional bend
  \subfigure[\emph{Bend-90°} con diseño tradicional]{\includegraphics[width=0.4\textwidth]{image/introduction/traditional-bend.png}}
  \hfill
  % Inverse design bend
  \subfigure[\emph{Bend-90°} obtenido con diseño inverso. Extraído de \citep{Su2020}]{
    \includegraphics[width=0.4\textwidth]{image/introduction/inverse-design-bend.png}
  }

  % 2° row
  % Traditional splitter
  \subfigure[\emph{2--splitter} con diseño tradicional]{\includegraphics[width=0.4\textwidth]{image/introduction/traditional-splitter.png}}
  \hfill
  % Inverse design splitter
  \subfigure[\emph{2-splitter°} obtenido con diseño inverso. Extraído de \citep{Su2020}]{
    \includegraphics[width=0.4\textwidth]{image/introduction/inverse-design-splitter.png}
  }

  \caption{Diseños tradicionales y obtenidos a partir de diseño inverso de un \emph{bend-90°} y \emph{2-splitter}}
  \label{fig:devices}

\end{figure}

\section{Descripción del  Problema}


Para poder calcular las propiedades físicas de un dispositivo (e.g. campo eléctrico, transmitancia) se debe resolver las ecuaciones de Maxwell.
Así, con el objetivo de evaluar las propiedades de cualquier geometría se suele utilizar métodos numéricos como elementos finitos (FEM) y diferencias finitas en el dominio de tiempo (FDTD) \citep{Schneider2019}.
Con estos planteamientos se selecciona una región rectangular a optimizar y se la divide  en $n \times m$  píxeles como si fuera una imagen, ver Figure \ref{fig:bend-discretization}. 
Luego, a cada píxel se le asocia el número $0$ o $1$.
Típicamente, cero representa la presencia de $SiO_2$ en la ubicación del píxel y uno la presencia de $Si$ \citep{Molesky2018}.

\begin{figure}[h]
  \centering
  \includegraphics[scale=0.6]{image/introduction/bend-discretization.png}
  \caption{\emph{Bend-90} con una región de diseño discretizada en $18 \times 18$ píxeles. Cada píxel negro representa la presencia de $Si$ y cada píxel blanco de $SiO_2$}
  \label{fig:bend-discretization}
\end{figure}

El diseño inverso comienza definiendo los requerimientos del dispositivo para luego tratar de buscar entre los $2^{n \times m}$ posibles diseños algún candidato que se adapte a lo que se busca \citep{Su2020, Molesky2018}.
Como prueba de concepto, trabajos como el de \cite{Malheiros-Silveira2020} parametrizaron $2^{10 \times 10}$ posibles diseños.
Por otro lado, \cite{Su2020} con el objetivo de fabricar un dispositivo con buen rendimiento, configuró un espacio de búsqueda de $2^{34 \times 34}$ opciones.
Así, se presentan algunas dificultades con el diseño inverso:

\begin{enumerate}
  \item Es imposible evaluar todas los posibles diseños.
  \item Las simulaciones computacionales son muy costosas \citep{Kudyshev2020}.
  \item El espacio de búsqueda es altamente no convexo \citep{Su2018}.
  \item No todos los diseños son fabricables \citep{Su2020}.
  \item Cada dispositivo es una clase distinta de problema \citep{Molesky2018}.
\end{enumerate}


De este modo, existe una demanda crítica de un \emph{framework} capaz de optimizar dispositivos con un elevado número de parámetros dentro de un espacio de búsqueda no convexo \citep{Kudyshev2020}. Este es un problema muy grande, por ello en la presente tesis nos centraremos en optimizar un \emph{bend-90°} y un \emph{2-splitter}.

\section{Justificación}

Deste el punto de vista industrial, el \emph{bend-90°} y \emph{2-splitter} se han escogido debido a que han sido estudiados usando diseño inverso desde el 2004, pero aún no han encontrado aplicación masiva en la industria \citep{Molesky2018}. 

Desde el punto de vista computacional, este problema es interesante porque ya hay estrategias computacionales conocidas para resolverlo, desde algoritmos evolutivos \citep{Hansen2016} hasta redes neuronales \citep{Goodfellow2015} y \emph{depth learning} \citep{Malkiel2018}. 
Además, debido al alto costo computacional de las simulaciones \citep{Schneider2019}, el trabajo requiere de computación de alto desempeño.
Así, es probable que se pueda obtener buenos resultados en la investigación aplicando el conocimiento ya existente en computación.

\section{Objetivos}

\begin{itemize}

  \item Evaluar y comparar el rendimiento y la convergencia de algoritmos de optimización usadas para optimizar un \emph{bend-90°} y \emph{2-splitter} usando computación de alto rendimiento.

  \item Fabricar el diseño con mejor rendimiento que se obtenga del \emph{bend-90°} y del \emph{2-splitter} para poder comparar las simulacionales computacionales con las mediciones físicas.

%El siguiente objetivo es implementar un modelo de \emph{deep learning} que pueda predecir el \emph{performance} de un \emph{bend-90°} y de un \emph{2-splitter} con un error cuadrático medio menor a \leonidas{TODO: INVESTIGAR CUANTO DEBERÍA SER}. 

\end{itemize}



\section{Aportes}

Este trabajo busca brindar una comparativa de las técnicas de optimización más relevantes que se aplican para optimizar un \emph{bend-90°} y un \emph{2-splitter} cuando estos son parametrizados con un elevado número de variables.

\chapter{Marco  Teórico}

El marco teórico deberá contener sólo la teoría necesaria para que el lector pueda comprender su propuesta. No sea muy detallista ni específico, intente resumir y colocar los conceptos más importantes.  (Entre 2 a 3 páginas)

%\chapter{Trabajos relacionados}

Usted deberá buscar, revisar, seleccionar y estudiar un conjunto de artículos  pertenecientes  al  estado del arte  y  que además, sean relevantes en su área de investigación. (Entre 2 a 3 páginas)\\


La sección de trabajo relacionados, deberá contener un resumen, claro y conciso, de aquellos trabajos de investigación que están directamente relacionados con vuestro trabajo. Esto con el objetivo que el lector pueda darse cuenta de las diferencias entre su propuesta y las del estado del arte, además, permitirá saber como otros autores han intentado resolver un problema similar al suyo.


%\chapter{Propuesta}

En la propuesta deberá colocar la o las ideas principales de lo que pretende realizar. Se sugiere  utilizar una figura a modo de \textit{pipeline} donde se muestre,  gráficamente, cada uno de los pasos que intervienen en su propuesta.  (4 a 5 páginas)


%\chapter{Resultados Preliminares}

En el presente capítulo se muestran resultados preliminares de la investigación.
Estos resultados se obtuvieron siguiendo la propuesta descrita en el anterior capítulo, pero llegando solo hasta la optimización continua.
Además, los experimentos se realizaron solo con un \emph{bend} utilizando un único algoritmo: CMA-ES.
Con las anteriores restricciones se realizaron tres experimentos.


\section{Experimento 1}

Como primer experimento se comenzó configurando un \emph{bend} con las características descritas en la figura \ref{fig:dimensiones-bend}.
Luego, se procedió a realizar la optimización.
Para la optimización continua se limitó la cantidad de simulaciones a 5000 y para cada una de las 5 etapas de la optimización discreta se limitó en 5000 simulaciones.

\begin{figure}[ht]
  \centering
  \includegraphics[width=\textwidth]{image/results/iterations_v1.png}
  \caption{Gráfica del número de simulaciones vs transmitancia producto de optimizar un \emph{bend} usando CMA-ES en el experimento 1.}
  \label{fig:iterations-v1}
\end{figure}

En la figura \ref{fig:iterations-v1} se muestra la transmitancia de cada uno de los diseños evaluados.
Podemos observar que en la etapa de la optimización continua, $\beta = 1$, es cuanto mayor crecimiento se logra.
Luego, en los distintos pasos dentro de la optimización discreta $(\beta = 2, 3, 4, 5, 6)$ el crecimiento es más lento.
Sin embargo, es destacable el hecho que aún al incrementar el valor de $\beta$, la simulación sigue logrando explorar diseños con valores de transmitancia similares.
Para lograr esto se comenzó ejecutando el algoritmo con un valor de $\sigma = 0.3$ en la optimización continua y luego se uso $\sigma = 0.01$ en la optimización discreta.


En total el experimento duró 24 horas con 30 minutos y en promedio cada simulación tomó 2.788 segundos.
Esto se debió a que se utilizó una resolución de 100 $nm$.
Pero, como el \emph{bend} que definimos utilizaba un tamaño de píxel de $60 nm$, las simulaciones no estaban logrando simular de forma correcta todos los detalles de los diseños.

\begin{figure}
  \centering
  \includegraphics[width=0.80\textwidth]{image/results/device-v1-r40.png}
  \caption{Intensidad de campo eléctrico y transmitancia del diseño obtenidos en el experimento 1 simulado bajo una resolución de 40 $nm$. La línea blanca ubicada en la parte inferior derecha indica la escala de 1 $um$ en la figura.}
  \label{fig:exp1}
\end{figure}

En la figura \ref{fig:exp1} se muestra el diseño obtenido con este experimento, pero simulado con una resolución de $40 nm$.
Como se puede observar, hay una reducción a practicamente la mitad de la transmitancia que se estaba obteniendo al usar una resolución de $100 nm$.

\section{Experimento 2}

Con el anterior experimento se logró determinar la importancia de utilizar una adecuada resolución. Así, se procedió a trabajar con $40 nm$ de resolución.

\begin{figure}[ht]
  \centering
  \includegraphics[width=\textwidth]{image/results/iterations_v2.png}
  \caption{Gráfica del número de simulaciones vs transmitancia producto de optimizar un \emph{bend} usando CMA-ES en el experimento 2.}
  \label{fig:iterations-v2}
\end{figure}


En total el experimento duró 4 días con 16 horas y en promedio cada simulación tomó 60.543 segundos.
Esto se debe al haber puesto la resolución en $40 nm$.

Como se puede observar en la figura \ref{fig:iterations-v2}, la optimización continua se realizó por alrededor de 3200 simulaciones y solo se avanzó un poco en la optimización discreta. Esto se debe al tiempo que estaba tomando la optimización. 
Sin embargo, aún cuando se realizaron menos simulaciones se obtuvieron mejores resultados que en el experimento 1 y con un mayor grado de confianza debido a la resolución usada.
En la figura \ref{fig:exp2-field-cont} se puede corroborar lo destacable de estos resultados.
Sin embargo, debido a la reducción en la cantidad de simulaciones, la optimización no logró llegar a fases con valores de $\beta$ mayores a 2, por ello los diseños aún no han convergido a valores reales.
Esto se puede apreciar en la figura $\ref{fig:exp2-eps-cont}$ donde las regiones grises representan la permitividad de un material no conocido.

\begin{figure}
  \centering
  \includegraphics[width=0.80\textwidth]{image/results/device-v2-r40-cont.png}
  \caption{Intensidad de campo eléctrico y transmitancia del diseño obtenidos en el experimento 2 simulado bajo una resolución de 40 $nm$.}
  \label{fig:exp2-field-cont}
\end{figure}

\begin{figure}
  \centering
  \includegraphics[width=0.60\textwidth]{image/results/eps-v2-cont.png}
  \caption{Distribución de la permitividad eléctrica del diseño del experimento 2.}
  \label{fig:exp2-eps-cont}
\end{figure}

Al resultado de este experimentó se le sometió a una discretización forzada (acercar la permitividad de cada píxel a la del $Si$ u $SiO_2$ de acuerdo a cual valor estuviera más cerca).
El diseño resultante se muestra en la figura \ref{fig:exp2-field-disc}, se observa que hay una reducción en el valor de la transmitancia.
A pesar de ello, los resultados son mejores que el experimento 1, esto nos lleva a suponer que de haber logrado ejecutarse la misma cantidad de iteraciones que en el experimento 1, se podría haber logrdo un dispositivo discretizado con una transmitancia alrededor del 95\%.

\begin{figure}
  \centering
  \includegraphics[width=0.80\textwidth]{image/results/device-v2-r40-disc.png}
  \caption{Intensidad de campo eléctrico y transmitancia del diseño obtenidos en el experimento 2 simulado bajo una resolución de 40 $nm$ tras ser discretizado.}
  \label{fig:exp2-field-disc}
\end{figure}


\section{Experimento 3}

Como tercer experimento se analizaron los resultados de las últimas evaluaciones realizadas en el anterior experimento.
Producto de ello, se encontró un diseño que ya estaba practicamente discretizado y que mantenía una transmitancia del 95.701 \%, además que presentaba curvas suaves que presumiblemente no tendrían dificultades para ser fabricadas. Este diseño se puede apreciar en la figura \ref{fig:exp3}.

\begin{figure}
  \centering
  \includegraphics[width=0.80\textwidth]{image/results/device.png}
  \caption{Intensidad de campo eléctrico y transmitancia del diseño obtenidos en el experimento 3 simulado bajo una resolución de 40 $nm$.}
  \label{fig:exp3}
\end{figure}


A partir de los experimentos realizados se observa que se está logrando el objetivo de la tesis.
Cuando se logre completar las siguientes etapas de la metodología, principalmente las simulaciones en 3D y restricciones de fabricación, parece razonable el suponer que se obtendrán resultados favorables.

%\input{section/conclusions}
%\recommendations{

Inicie aquí el párrafo de recomendaciones


}



%% ============================================================================
\renewcommand{\bibname}{\large\bf{REFERENCIAS BIBLIOGRÁFICAS}}
\bibliographystyle{apalike-etal-in-italics} % Use the IEEEtran style
\bibliography{references}   % The references are in "referencias.bib"

\end{document}
%% ----------------------------------------------------------------


%%% Local Variables:
%%% mode: latex
%%% TeX-master: t
%%% End:
%%% End:
