\chapter{Conclusiones y Trabajos Futuros}\label{chapter:conclution}

\section{Conclusiones}

\begin{itemize}

  \item Se logró diseñar tanto un \emph{bend} como un WDM con transmitancias mayores al 90 \% y
        con diseños resilientes a errores de dilatación y erosión.
        Además, el \emph{bend} optimizado logra funcionar con una transmitancia de alrededor de 90 \%
        en el ranga de $1500nm$ a $1600 nm$. Por otro lado, el WDM optimizado muestra un comportamiento
        sin cambios bruscos al trabajar entre $1250 nm$ a $1600 nm$.

  \item La optimización topológica facilitó la búsqueda de los diseños con las propiedades deseadas.
        Sin embargo, parece imprescindible el uso de algoritmos de primer orden al utilizar
        esta estrategia. Caso contrario, la convergencia a diseños compactos avanza a un ritmo muy lento.

  \item Para el caso del \emph{bend} definir la función objetivo como maximizar la transmitancia en una
        sola longitud de onda ha permitido obtener un diseño que funciona bien incluso en longitudes de
        onda cercanas. Por otro lado, la función objetivo empleada para el WDM se estancó en todos los
        casos (excepto en uno) en diseños que no lograban conectar de forma funcional ambas guías de salida.
        Así, parece sensato afirmar que es necesario buscar otra alternativa para definir esta función
objetivo.

  \item Para la optimización de un \emph{bend} y WDM con una región de diseño de $2 \mu m \times 2 \mu m$,
        el algoritmo L-BFGS-B ha mostrado superioridad en términos de convergencia y desempeño con respecto
        al G-CMA-ES, MMA, G-PSO y G-GA. 

  \item Los algoritmos G-PSO y G-GA mostraron resultados similares y muy superior a sus contrapartes que no
        usan la información de la gradiente (PSO y GA).
        Así, parece necesario explorar el desempeño que estos algoritmos en trabajos similares
        donde se han aplicado simplemente PSO y GA.

  \item El algoritmo G-CMA-ES ha mostrado mejores resultados que su contraparte que no usa la información
        de la gradiente. Sin embargo, la convergencia de este algoritmo es más lenta que otras alternativas
        como L-BFGS-B, G-PSO y G-GA. Además, los diseños conseguidos poserían regiones dificiles de fabricar.
        De este modo, el algoritmo no parece ser la mejor opción a utilizar en optimización topológica.

\end{itemize}

\section{Trabajos Futuros}

Próximamente, se espera concretar la fabricación de los diseños optimizados.
Además, parece interesante estudiar otras restricciones que se puedan aplicar a la función
objetivo del WDM con el fin de asegurar conectividad.
