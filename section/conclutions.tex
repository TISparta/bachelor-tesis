\chapter{Conclusiones y Trabajos Futuros}\label{chapter:conclutions}

\section{Conclusiones}

\begin{itemize}

  \item Siguiendo la estrategia planteada, se logra diseñar tanto un \emph{bend} 
        como un WDM con transmitancias mayores al 90 \% y
        con diseños resilientes a errores de dilatación y erosión.
        Además, el \emph{bend} optimizado logra funcionar con una transmitancia de alrededor de 90 \%
        en el rango de $1500nm$ a $1600 nm$. 
        Por otro lado, el WDM optimizado muestra un comportamiento
        sin cambios bruscos al trabajar entre $1250 nm$ a $1600 nm$.

  \item La optimización topológica facilita la búsqueda de los diseños robustos
        ante errores de fabricación.
        Sin embargo, parece imprescindible el uso de algoritmos de primer orden al utilizar
        esta estrategia. 
        Caso contrario, la convergencia a diseños compactos avanza a un ritmo muy lento.

  \item Para el caso del \emph{bend}, definir la función objetivo 
        como maximizar la transmitancia en una sola longitud de onda
        permite obtener diseños que funcionan bien incluso en longitudes de
        onda cercanas. 
        Por otro lado, la función objetivo empleada para el WDM 
        logra producir buenos resultados, mas en promedio esto no es así.
        Es necesario incorporar un término que incentive que el diseño
        termine con todas las guías de onda conectadas.

  \item Para la optimización de un \emph{bend} y WDM con una región de diseño de $2 \mu m \times 2 \mu m$,
        el algoritmo MMA converge rápidamente y se estanca en diseños no funcionales y
        difíciles de fabricar. Así, parece necesario un mayor estudio al aplicar
        este algoritmo en estos dos casos.

  \item Excluyendo los resultados del MMA, el algoritmo L-BFGS-B ha mostrado superioridad
        en términos de desempeño, convergencia y porcentaje de regiones grises. 
        Parece sensato comenzar futuras investigaciones aplicando este algoritmo.

  \item El algoritmo G-CMA-ES ha mostrado mejores resultados que su contraparte que no usa
        la información de la gradiente. Sin embargo, posee la peor convergencia de los
        algoritmos estudiados, bajos valores del FOM y 
        produce diseños no correctamente binarizados.
        De este modo, el algoritmo no parece ser la mejor opción a utilizar en optimización
        topológica. 
        
  \item En promedio, el algoritmo G-PSO muestra mejores resultados que G-GA en
        términos de desempeño y porcentaje de regiones grises.
        Además, ambos algoritmos muestran un desempeño exponencialmente mejor
        a sus contrapartes que no usan la información de la gradiente.

\end{itemize}

\section{Trabajos Futuros}

Próximamente, se espera concretar la fabricación de los diseños optimizados.
Además, parece interesante estudiar otras restricciones que se puedan aplicar a la función
objetivo del WDM con el fin de asegurar conectividad.
