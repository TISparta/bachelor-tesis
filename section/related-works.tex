\chapter{Trabajos relacionados}

En el presente capítulo se resume los puntos principales de trabajos relacionados al problema de investigación.
Para ello dividiremos el contenido en tres secciones. 
En primer lugar, mostraremos trabajos que han usado parametrización por píxeles.
En segundo lugar, veremos investigaciones que utilizan parametrización por conjuntos de nivel.
En tercer lugar, describiremos publicaciones que emplearon parametrización por segmentos.

\section{Parametrización por píxeles}

\cite{Su2020} divide una región cuadrada de $2.5 \mu m \times 2.5 \mu m$ en píxeles de $40 nm \times 40 nm$ y luego lo une a guías de onda de $40 nm$ para diseñar un \emph{bend-90°}. 
La definición de su función objetivo es como la descrita en el marco teórico, pero además muestra ideas de como definir esta función de manera que la optimización intente acercarse a un valor de transmitancia deseado.

Como algoritmo de optimización se utiliza $L-BFGS-B$ y $MMA$ ejecutando 100 iteraciones en la etapa continua y 180 en la discreta.
El proceso completo de optimización se realizó en SPINS demorando catorce horas y media al ser ejecutado en una computadora con 5 GPU, 2 CPU con 64 GB de RAM.
Sin embargo, solo utilizó menos de 4 GB de RAM y cada simulación demoró alrededor de un minuto.

Para asegurar restricciones de fabricación, \cite{Su2020} se aseguró que el tamaño de cada píxel sea mayor al mínimo tamaño que se podía fabricar.

\section{Parametrización por conjuntos de nivel}

\cite{Piggott2017} diseña un \emph{3-splitter} dividiendo una región de $3.8 \mu m \times 2.5 \mu m$ en píxeles de $40 nm \times 40 nm$.
En la definición de su función objetivo busca trabajar con 3 longitudes de onda y que con cada longitud la transmitancia se maximice en una guía de onda de salida y se minimice en las otras, similar al trabajo de \cite{Su2020} en la optimización de un \emph{2-splitter}.

Luego, utilizando SPINS como simulador, realiza la optimización con un algoritmo propio que está basado en \emph{gradient-descent} y \emph{line search}. 
Con este algoritmo se encarga de incorporar restricciones del mínimo radio de curvatura que puede contener el diseño y limita el tamaño máximo que pueden tener los huecos que se formen en el dispositivo.

\section{Parametrización por segmentos}


\cite{Prosopio-Galarza2019} utiliza esta estrategia para dividir un rectángulo de $2\mu m \times 1.5 \mu m$ en 13 rectángulos verticales idénticos que particionan la región de diseño.
Luego, esto es unido a tres guías de onda fijas de $0.5 \mu m$ para formar un \emph{2-splitter} de un esperor de $0.2 \mu m$.


Como función objetivo se establece maximizar la transmitancia. Seguidamente, por separado utiliza tres algoritmos para optimizar el diseño: i) \emph{Particle Swarm Optimization}, ii) \emph{Shrinking Box algorithm}, iii) \emph{Steepest Ascent algorithm}. 

La simulación se realiza usando ANSY Lumerical FDTD. 
Los experimentos solo se repitieron una vez y se ejecutó cada algoritmo durante 30 iteraciones.
Probablemente, el usar este simulador mediante la interfaz gráfica limitó la automatización de los experimentos para conseguir una mayor cantidad de datos a comparar. 
Pero, es destacable que en su investigación \cite{Prosopio-Galarza2019} logra determinar que los ángulo más agudos, los cuales son los más adecuados como regla práctica de diseño, son formados con la optimización mediante \emph{Particle Swarm Optimization}.

