\chapter*{\center \Large \vspace{-5.5cm} ABSTRACT}
\addcontentsline{toc}{section}{\bfseries ABSTRACT}
\markboth{ABSTRACT}{ABSTRACT} 

\begin{center}
\Large \vspace{-1.5cm} \textbf{Evaluation of First-Order Optimization Algorithms in Robust Topology Optimization of Nanophotonic Devices}
\end{center}

Silicon photonics is an emerging area with constant growth in the last decades.
Photonic devices show potential applications to improve computing systems, telecommunications and other areas.
Nevertheless, it is still a challenge to integrate a great number of fundamental photonic
devices in a chip with small area and low loss.
In this work we designed two fundamental photonic devices: (i) bend and (ii)
2-channel wavelength-demultiplexer (WDM).
The designs were done on a $2 \mu m \times 2 \mu m$ area following a robust topology
optimization based strategy.

We evaluated and comparated five first-order optimization algorithms:
(i) \emph{Limited-memory Broyden–Fletcher–Goldfarb–Shanno with boundaries} (L-BFGS-B), 
(ii) \emph{Method of Moving asymptotes} (MMA), 
(iii) \emph{Covariance Matrix Adapatation Evolution Strategy} (G-CMA-ES), (iv) \emph{Gradient Particle Swarm
Optimization} (G-PSO) and (v) \emph{Gradient Genetic Algorithm} (G-GA). 
The last three algorithms are variations of their more standard versions (CMA-ES, PSO and GA)
where the computation of the gradient is included to guide the optimization process.

The best optimized designs show:
(i) transmission greater than $90 \%$ and robustness to over/under-etching,
(ii) a gray percentage of less than $2 \%$ and
(iii) their performance is broadband consistent with smooth changes 
(1500-1600 nm \emph{bend} and 1250-1600 nm WDM) even after deleting non-convex regions.

These results are promising for 
(i) the integration of WDM devices in an lower area than 
state of the art (< $2.8 \mu m \times 2.8 \mu m$) and 
(ii) the design of bends with lower loss than intuitive-traditional designs of $1 \mu m$ radius.

\noindent \textbf{Keywords:}\\
\noindent Optimization Algorithms; Inverse Design; Integrated Photonic; Numeric Methods; Robust Topology
Optimization.

