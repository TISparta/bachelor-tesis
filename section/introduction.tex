\chapter{Motivación y Contexto}\label{chapter:introduction}

\section{Introducción}

La fotónica es la ciencia que estudia la generación, detección y manipulación de la luz. 
Los principales beneficios que ofrece son \citep{Shen2019}:
(i) elevado ancho de banda en comunicaciones, 
(ii) bajo consumo energético,
(iii) interconexiones ópticas independientes de la distancia.
Actualmente, existen diversas aplicaciones que aprovechan estos beneficios, por ejemplo:
(i) interconexiones ópticas en centrales de datos \citep{Shen2019},
(ii) redes neuronales ópticas \citep{Shen2017} e
(iii) internet de las cosas \citep{Li2021}.


\leonidas{A partir del proyecto Top500, un problema encontrado en los sistemas de computación de alto
desempeño (HPC, por sus siglas en Inglés)}
es el ratio entre el ancho de banda entre nodos y el poder de procesamiento por nodo ($byte / FLOP$).
Este valor ha decrecido en un factor de seis en los últimos años.
Es decir, la capacidad para interconectar nodos está
limitando el desempeño de sistemas HPC en programas que hacen uso intensivo de transferencias de datos.
Ante este problema, los avances en la fotónica en silicio (SiP) integrada se presenta como una de las
principales alternativas de solución porque permiten implementar interconexiones a distancias del orden de metros,
manteniendo un elevado ancho de banda y bajo consumo energético \citep{Shen2019, Anderson2018}.

El diseño e integración de dispositivos SiP, en términos
de cantidad de dispositivos por chip, se encuentra aún en una etapa inicial
\citep{LukasChrostowski2010, Glick2018}.
Sin embargo, ya existen procesos de fabricación estándar en \emph{foundries}
para fabricar chips SiP a un precio accesible, compatibles con procesos CMOS
a través de \emph{process design kits} (PDKs) \citep{Bogaerts2018}.

La alta densidad en la fabricación de dispositivos SiP integrados es un desafío porque requiere mantener bajas pérdidas de energía en el chip a nivel de sistema fotónico. Esto motiva la optimización de dispositivos fundamentales en sistemas fotónicos \citep{Vuckovic2019}, tales como 
 \emph{multi-channel wavelength-demultiplexer}, \emph{grating couplers}, etc.
Para esto existen dos estrategias principales: (i) diseño tradicional
\citep{Hughes2016, Song2008, Huang2018} y (ii) diseño inverso \citep{Malheiros-Silveira2020, Gregory2015, Su2020}.


\begin{figure}[ht]
  \centering
  % 1° row
  % Traditional bend
  \subfigure[\emph{Bend} con diseño
  tradicional.]{\includegraphics[width=0.4\textwidth]{image/introduction/traditional-bend.png}}
  \hfill
  % Inverse design bend
  \subfigure[\emph{Bend} obtenido con diseño inverso. Extraído de
  \citep{Su2020}.]{
    \includegraphics[width=0.4\textwidth]{image/introduction/inverse-design-bend.png}
  }

  % 2° row
  % Traditional splitter
  \subfigure[\emph{Splitter} con diseño tradicional.]{\includegraphics[width=0.4\textwidth]{image/introduction/traditional-splitter.png}}
  \hfill
  % Inverse design splitter
  \subfigure[WDM obtenido con diseño inverso. Extraído de \citep{Su2020}.]{
    \includegraphics[width=0.4\textwidth]{image/introduction/inverse-design-splitter.png}
  }

  \caption{Diseños tradicionales y obtenidos a partir de diseño inverso de un
  \emph{bend} y un WDM. 
  En todas las figuras la luz ingresa por la guía de onda izquierda y se propaga hacia la derecha.}
  \label{fig:devices}

\end{figure}

La \autoref{fig:devices} presenta una comparación de dos metodologías de diseño: tradicional e inverso.
Se observa que en el diseño tradicional (izquierda) se define el dispositivo con geometrías simples que permiten obtener funciones analíticas de sus propiedades físicas \citep{Hughes2016, Song2008}. 
Esto se realiza para poder optimizar la función obtenida a partir de los parámetros que la definan. 
Dicha optimización se suele ejecutar haciendo un barrido de los parámetros, con algoritmos genéticos o usando \emph{particle swarm optimization}.
Para obtener buenos resultados con esta metodología es requerido basarse en la habilidad y confiar en la intuición por parte del diseñador \citep{Su2020}. 

Existen tres grandes inconvenientes con el diseño tradicional. 
Primero, solamente se explora una pequeña fracción de todos los posibles diseños.
Segundo, por lo general no es conocido el límite de rendimiento del dispositivo
\citep{Molesky2018}.
Tercero, al trabajar en la escala de nanómetros, existen casos como el
\emph{bend-90°} y \emph{wavelength-demultiplexer} que presentan un bajo rendimiento con diseños tradicionales \citep{Su2020}.

Por otro lado, en el diseño inverso, mostrado en la \autoref{fig:devices}
(derecha), las geometrías resultantes no están limitadas a diseños intuitivos o regulares.
Esta metodología consiste en primero definir las propiedades deseadas en nuestro dispositivo (e.g., transmitancia > 90\%, nivel de gris < 2 \%, etc).
Luego, usando simulaciones computacionales podemos determinar las propiedades físicas
de una geometría arbitraria. El problema de diseño se reduce a parametrizar una región
de diseño y explorar distintas geometrías con esta parametrización hasta encontrar un diseño
que cumpla las propiedades deseadas.
Esto se formula como un problema de optimización, para lo cual existe una variedad de algoritmos
que se pueden aplicar (e.g., \emph{genetic algorithms}, \emph{particle swarm optimization}, etc)
\citep{Molesky2018, Su2020}.
\leonidas{En \cite{Schneider2019, Elsawy2020, Campbell2019} se puede encontrar una descripción detallada 
de los distintos algoritmos de optimización
comúnmente empleados en el diseño de dispositivos fotónicos}.


El diseño inverso ha logrado conseguir dispositivos con menores pérdidas que las obtenidas por el
diseño tradicional por lo que ha ganado interés en el área de fotónica durante
las últimas dos décadas \citep{Su2018, Molesky2018, Campbell2019}. 
Identificamos que existen cuatro grandes desafíos con este planteamiento:
(i) el espacio de búsqueda es exponencial \citep{Vuckovic2019}, 
(ii) las simulaciones computacionales son costosas en términos de tiempo y memoria \citep{Kudyshev2020}, 
(iii) el espacio de búsqueda es no convexo \citep{Su2018} y
(iv) no todos los diseños son fabricables \citep{Su2020}.


El presente trabajo se centra en estudiar dos dispositivos SiP
fundamentales:
(i) \emph{bend-90°} y (ii) \emph{2-channel wavelength demultiplexer}.
De aquí en adelante nos referiremos a estos simplemente como \emph{bend} y WDM.
Así, el objetivo de esta tesis es aplicar el conocimiento en computación para
encontrar diseños de estos dispositivos con eficiencias mayores al 90\%
y robustos ante errores de fabricación trabajando en la escala de nanómetros. 
Empleamos el diseño inverso para encontrar los diseños que cumplen con estas propiedades y
realizamos una parametrización topológica para explorar los posibles diseños.
El estudio se desarrolló con cinco algoritmos de optimización: (i) \emph{Limited-memory Broyden–Fletcher–Goldfarb–Shanno with boundaries} (L-BFGS-B), 
(ii) \emph{Method of Moving asymptotes} (MMA), 
(iii) \emph{Covariance Matrix Adapatation Evolution Strategy} (G-CMA-ES), (iv) \emph{Gradient Particle Swarm Optimization} (G-PSO) y (v) \emph{Gradient Genetic Algorithm} (G-GA).


El presente documento está organizado de la siguiente manera:

El \autoref{chapter:introduction}  brinda una introducción al tema de investigación, describe el problema a detalle, justifica la relevancia de resolverlo, define los objetivos y señala los aportes del trabajo.

El \autoref{chapter:theory} desarrolla los conceptos necesarios para entender la propuesta presentada
en esta tesis.

En el \autoref{chapter:related-works} presentamos una revisión del estado del arte centrándose
en algoritmos utilizados en el diseño inverso y dos estrategias populares de parametrización:
(i) basada en conjuntos de nivel y (ii) basada en píxeles.


En el \autoref{chapter:methodology} presentamos nuestra metodología propuesta para el diseño inverso de dispositivos \emph{bend} y WDM usando optimización topológica robusta. 
Brindamos detalles sobre la configuración
de las simulaciones y algoritmos (dimensiones, parámetros, etc).
Además, se describe las etapas de optimización seguidas y el posprocesamiento realizado a los diseños mejor optimizados.


El \autoref{chapter:results} muestra los resultados obtenidos por  experimentación y la discusión sobre estos.
Comparamos los resultados de los cinco algoritmos evaluados en base a tres criterios: 
(i) convergencia, (ii) eficiencia y (iii) porcentaje de región gris.
Adicionalmente, se analiza y compara con el estado del arte nuestros diseños mejor optimizados
del \emph{bend} y WDM.

Por último, en el \autoref{chapter:conclutions} se detallan las conclusiones del presente trabajo de investigación.

\section{Descripción del Problema}

\leonidas{Una forma de definir parámetros a partir de los cuales se pueda representar un dispositivo 
(i.e. parametrizar)} es seleccionando una región rectangular y dividiéndola
en $n \times m$ píxeles como si fuera una imagen, ver \autoref{fig:bend-discretization}.
Luego, cada píxel se rellena con dos posibles materiales: óxido de silicio $(SiO_2)$ o silicio ($Si$).
Finalmente, se usa métodos numéricos para resolver las ecuaciones de Maxwell en este diseño
y obtener así la distribución del campo eléctrico, este permite entender el funcionamiento del dispositivo
y sus propiedades \citep{Molesky2018, Schneider2019}.

\begin{figure}[ht]
  \centering
  \includegraphics[scale=0.6]{image/introduction/bend-discretization.png}
  \caption{\emph{Bend} con una región de diseño discretizada en $18 \times 18$
  píxeles. Cada píxel negro representa la presencia de $Si$ y cada píxel blanco
  de $SiO_2$.}
  \label{fig:bend-discretization}
\end{figure}

Con esta parametrización, el diseño inverso comienza definiendo los requerimientos del dispositivo para luego tratar de buscar entre los $2^{n \times m}$ posibles diseños algún candidato que se adapte a lo que se busca \citep{Su2020, Molesky2018}.
Como prueba de concepto, trabajos como el de \cite{Malheiros-Silveira2020} parametrizaron $2^{10 \times 10}$ posibles geometrías.
Pero, se presentan algunas dificultades con esta estrategia:

\begin{enumerate}
  \item No es viable evaluar todos los posibles diseños por haber un número excesivamente elevado de ellos \citep{Vuckovic2019}.
  \item Las simulaciones computacionales son muy costosas en términos de consumo de memoria y 
    tiempo de ejecución \citep{Kudyshev2020}.
  \item El espacio de búsqueda es no convexo \citep{Su2018}.
  \item No todos los diseños son fabricables por limitaciones físicas \citep{Su2020}.
  \item Cada dispositivo es una clase distinta de problema, es decir, no necesariamente funcionará la misma estrategia para cada dispositivo \citep{Molesky2018}.
\end{enumerate}

Además, la fabricación viene con otros desafíos, principalmente:

\begin{enumerate}
  \item Errores de precisión en los instrumentos \citep{Piggott2017}.
  \item Sensibilidad ante cambios de temperatura \citep{Vuckovic2019}.
\end{enumerate}

\leonidas{Considerando las anteriores dificultades, el problema es encontrar una parametrización 
que genere un dispositivo que optimice alguna propiedad deseada, calculado mediante simulaciones 
computacionales, y que pueda asegurar mantener un óptimo funcionamiento al ser fabricado.
Este problema se estudió para dos dispositivos nanofotónicos (i) \emph{bend} y (ii) WDM.}

\section{Justificación}


Las mejoras de estos dispositivos permiten diseñar \emph{hardware} eficiente y robusto para interconexión
óptica en sistemas de cómputo, lo cual permitirá mejorar la ejecución de programas en sistemas HPC. Esto
potencialmente beneficia aplicaciones de alto impacto \citep{Shen2019} tales como inteligencia artificial, analítica de datos y computación científica.

Desde el punto de vista computacional, 
este problema es interesante porque las estrategias computacionales y algoritmos conocidos para resolverlo
(e.g., algoritmos genéticos \citep{Mykel2019}, estrategias evolutivas \citep{Hansen2016}, etc.) motivan su
comparación para identificar sus limitaciones.


Además, debido al alto costo computacional de las simulaciones \citep{Schneider2019}, nuestro trabajo requiere de computación de alto desempeño para el procesamiento.

\section{Objetivos}

Definimos dos objetivos principales:

\begin{enumerate}

  \item Diseñar un \emph{bend} y WDM con eficiencias mayores al 90\% y robustos ante errores de fabricación en un área de diseño de $2 \mu m \times 2 \mu m$. 

  \begin{enumerate}

    \item Seleccionar una estrategia de parametrización que asegure facilidad de fabricación.

    \item Definir una función objetivo que encapsule las propiedades buscadas en cada dispositivo.

    \item Encontrar geometrías con valores óptimos de la función objetivo en simulaciones computacionales.

    \item Encontrar geometrías robustos ante posibles errores de fabricación de dilatación o erosión.

  \end{enumerate}

  \item Comparar el desempeño y la convergencia de cinco algoritmos de optimización populares usados para optimizar dispositivos nanofotónicos.

\end{enumerate}
