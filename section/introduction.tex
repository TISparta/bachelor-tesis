\chapter{Introducción}
%\intro{

\section{Motivación y Contexto}

La fotónica está atrayendo el interés de la industria debido a su potencial en términos de escalabilidad y los beneficios de costo-eficiencia. 
Este potencial se evidencia, por ejemplo, con los siguientes tres puntos. 
Primero, mantener la tendencia que cada 10 años se mejore en un factor de 1000 el \emph{performance} de los sistemas electrónicos parece ser posible solo con la convergencia de estos con sistemas fotónicos \cite{Glick2018}. 
Segundo, existe una inversión billonaria en la fabricación de transistores cuyos procesos se están comenzando a lograr adaptar para fabricar circuitos fotónicos \cite{LukasChrostowski2010}.
Tercero, el elevado ancho de banda que ofrece en comunicaciones digitales y el éxito de los cables ópticos \cite{LukasChrostowski2010, Glick2018}.
De esta manera, es prometedor el futuro de los dispositivos fotónicos.

Los dispositivos fotónicos se utilizan en grandes cantidades en los circuitos fotónicos integrados \cite{LukasChrostowski2010}. 
Estos dispositivos trabajan en la escala de nanómetros y son diseñados para funcionar bajo ciertos requerimientos. 
Así, para que estos dispositivos cumplan los requerimientos deseados existen dos estrategias comunes: diseño tradicional y diseño inverso \cite{Molesky2018}.


En el diseño tradicional se define el dispositivo con geometrías simples que permiten obtener funciones analíticas de sus propiedades físicas. 
Esto se realiza para poder optimizar la función obtenida a partir de los parámetros que la definan. Dicha optimización se suele realizar haciendo
un barrido de los parámetros, con algoritmos genéticos o usando \emph{particle swarm optimization}. Es un enfoque simple, pero que ha obtenido
buenos resultados. Sin embargo, existen dos grandes inconvenientes con este planteamiento. 
Primero, solo estamos explorando una pequeña fracción de todos los posibles diseños.
Segundo, por lo general no es conocido el límite del \emph{performance} del dispositivo \cite{Molesky2018, Su2020}.


En el diseño inverso se busca hacer una mayor exploración de todos los posibles diseños. 
Para ello, ya no nos limitamos a solo usar diseños intuitivos. Ahora, definimos geometrías arbitrarias y usamos simulaciones computacionales para determinar las propiedades físicas del dispositio \cite{Molesky2018, Su2020}. Este enfoque ha logrado conseguir mejores resultados que los obtenidos por el diseño tradicional en una variedad de dispositivos \cite{Su2018, Molesky2018}. Sin embargo, este planteamiento viene acompañado de nuevos desafíos.


\section{Descripción del  Problema}

Una estrategia común a la hora de hacer diseño inverso es seleccionar una región rectangular de diseño y dividirla en $n \times m$ rectángulos
de igual tamaño distribuidos como una matrix. 
Luego, seleccionamos dos materiales y decimos que cada rectángulo puede ser de alguno de ellos. 
De esta forma tenemos $2^{n \times m}$ posibles dispositivos \cite{Su2020}. 
Definiendo un dispositivo de esta manera podemos llegar a tener, por ejemplo, $2^{10 \times 10}$ posibles diseños \cite{Malheiros-Silveira2020}.
Pero, es imposible evaluar todas esas posibilidades.

\section{Justificación}
En este apartado, el lector debe entender por qué es importante resolver el problema que plantean, desde el punto de vista social y especialmente, computacional. 

La idea es  justificar la razón por la cual  el problema que intenta resolver es importante y relevante. 
Tenga en cuenta que el problema que intenta resolver puede ser de tipo aplicativo, y  en estos casos, se intenta aplicar algoritmos, métodos o técnicas para solucionar algún problema de otra área como
biología, medicina, sociología, entre otros. \\

Por ejemplo, \textbf{Detección de covid19 en imágenes de resonancia magnética   mediante técnicas de \textit{deep learning}} es un tema netamente aplicativo y la justificación del problema será  más del tipo social. Sin embargo, si se plantea una nueva arquitectura de CNN que mejore el rendimiento del estado del arte para, específicamente, detección de covid, podemos decir que la justificación iría tanto desde el punto de vista social como de ciencia de la computación \\


Por otro lado, si el problema que intenta resolver, es específicamente, de ciencia de la computación, como por ejemplo, mejorar alguna estructura de datos, modificar algún algoritmo para optimizar su eficiencia en RAM o velocidad, crear una nueva función de activación en el caso de redes neuronales, etc; entonces, la tesis está mas relacionada a ciencias básicas y por lo tanto la justificación será más desde el punto de ciencia de la computación.  \\

Es importante determinar el tipo de investigación que está desarrollando, para según esto redactar la justificación.

\section{Objetivos}
Deberá redactar el objetivo general y los específicos de la tesis. Recuerde, que el objetivo general debe estar en concordancia con el título de la tesis.


\section{Aportes}
Los aportes permiten visualizar la calidad de su trabajo. En general, para las tesis de pregrado no se piden trabajos que generen nuevo conocimiento, pero es posible realizar trabajos que innoven en alguna medida  mínima lo cual hace que su trabajo sea publicable.


%}
