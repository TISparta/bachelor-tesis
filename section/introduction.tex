\chapter{Introducción}
%\intro{

\cite{citacion1}

La introducción es una parte muy importante en todo documento en general, y por supuesto, en un trabajo de investigación. El objetivo de la introducción es hacer que el lector tenga una visión general de la tesis, desde el problema que intenta atacar, el porqué es relevante, como han intentado resolver el problema otros autores y cual o cuales son las diferencias principales con su propuesta, como usted propone resolver el problema,  e incluso los resultados más relevantes y el aporte del trabajo de investigación. \\

Recuerde que una buena introducción permite dar, al lector, un panorama general, pero completo, de todo el trabajo de investigación. (Máximo 2 hojas)

\section{Motivación y Contexto}


\section{Descripción del  problema}
En este apartado se espera que, mediante su redacción,   el lector pueda  comprenda, plenamente, cual es el problema que intenta resolver. Debe tener especial cuidado en la escritura, frases, oraciones y conectores pues es aquí donde el lector  se debe hace  una idea clara  del problema que usted intenta atacar.   

\section{Justificación}
En este apartado, el lector debe entender por qué es importante resolver el problema que plantean, desde el punto de vista social y especialmente, computacional. 

La idea es  justificar la razón por la cual  el problema que intenta resolver es importante y relevante. 
Tenga en cuenta que el problema que intenta resolver puede ser de tipo aplicativo, y  en estos casos, se intenta aplicar algoritmos, métodos o técnicas para solucionar algún problema de otra área como
biología, medicina, sociología, entre otros. \\

Por ejemplo, \textbf{Detección de covid19 en imágenes de resonancia magnética   mediante técnicas de \textit{deep learning}} es un tema netamente aplicativo y la justificación del problema será  más del tipo social. Sin embargo, si se plantea una nueva arquitectura de CNN que mejore el rendimiento del estado del arte para, específicamente, detección de covid, podemos decir que la justificación iría tanto desde el punto de vista social como de ciencia de la computación \\


Por otro lado, si el problema que intenta resolver, es específicamente, de ciencia de la computación, como por ejemplo, mejorar alguna estructura de datos, modificar algún algoritmo para optimizar su eficiencia en RAM o velocidad, crear una nueva función de activación en el caso de redes neuronales, etc; entonces, la tesis está mas relacionada a ciencias básicas y por lo tanto la justificación será más desde el punto de ciencia de la computación.  \\

Es importante determinar el tipo de investigación que está desarrollando, para según esto redactar la justificación.

\section{Objetivos}
Deberá redactar el objetivo general y los específicos de la tesis. Recuerde, que el objetivo general debe estar en concordancia con el título de la tesis.


\section{Aportes}
Los aportes permiten visualizar la calidad de su trabajo. En general, para las tesis de pregrado no se piden trabajos que generen nuevo conocimiento, pero es posible realizar trabajos que innoven en alguna medida  mínima lo cual hace que su trabajo sea publicable.


%}
