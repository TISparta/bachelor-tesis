\chapter{Motivación y Contexto}

\section{Introducción}

La fotónica está atrayendo el interés de la industria debido a su potencial en términos de escalabilidad y los beneficios de costo-eficiencia. 
Este potencial es evidente, por ejemplo, con los siguientes tres puntos. 
Primero, si se quiere mantener la tendencia que cada 10 años se mejore en un factor de 1000 el \emph{performance} de los sistemas electrónicos, entonces parece ser indispensable la convergencia de estos con sistemas fotónicos \citep{Glick2018}. 
Segundo, existe una inversión billonaria en la fabricación de transistores cuyos procesos se están comenzando a lograr adaptar para fabricar circuitos fotónicos \citep{LukasChrostowski2010}.
Tercero, el elevado ancho de banda que ofrece en comunicaciones digitales y el éxito de los cables ópticos \citep{LukasChrostowski2010, Glick2018}.

Los dispositivos fotónicos se utilizan en grandes cantidades en los circuitos fotónicos integrados \citep{LukasChrostowski2010}. 
Estos trabajan en la escala de nanómetros y son diseñados para funcionar bajo ciertas especificaciones. 
Así, para que estos dispositivos cumplan los requerimientos deseados existen dos estrategias comunes: diseño tradicional y diseño inverso \citep{Molesky2018}.


En el diseño tradicional se define el dispositivo con geometrías simples que permiten obtener funciones analíticas de sus propiedades físicas. 
Esto se realiza para poder optimizar la función obtenida a partir de los parámetros que la definan. Dicha optimización se suele realizar haciendo
un barrido de los parámetros, con algoritmos genéticos o usando \emph{particle swarm optimization}. Es un enfoque simple, pero que ha obtenido
buenos resultados. Sin embargo, existen dos grandes inconvenientes con este planteamiento. 
Primero, solo estamos explorando una pequeña fracción de todos los posibles diseños.
Segundo, por lo general no es conocido el límite del \emph{performance} del dispositivo \citep{Molesky2018, Su2020}.


En el diseño inverso se busca hacer una mayor exploración de todos los posibles diseños. 
Para ello, ya no nos limitamos a solo usar diseños intuitivos. Ahora, definimos geometrías arbitrarias y usamos simulaciones computacionales para determinar las propiedades físicas del dispositivo \citep{Molesky2018, Su2020}. Este enfoque ha logrado conseguir mejores resultados que los obtenidos por el diseño tradicional \citep{Su2018, Molesky2018}. Sin embargo, este planteamiento viene acompañado de nuevos desafíos.


\section{Descripción del  Problema}

Una estrategia común en diseño inverso es seleccionar una región rectangular a optimizar y dividirla en $n \times m$ \emph{
  pixels} como si fuera una imagen. 
Luego, a cada \emph{pixel} se le asocia el número $0$ o $1$. 
Cero representa la presencia de $SiO_2$ en la ubicación del \emph{pixel} y uno la presencia de $Si$.
De esta forma tenemos $2^{n \times m}$ posibles dispositivos \citep{Su2020}. 
Con esta definición podemos llegar a tener, por ejemplo, $2^{10 \times 10}$ posibles diseños \citep{Malheiros-Silveira2020}.
Así, surgen algunas dificultades con esta estrategia:

\begin{enumerate}
  \item Es imposible evaluar todas los posibles diseños.
  \item Las simulaciones computacionales son muy costosas \citep{Kudyshev2020}.
  \item El espacio de búsqueda es altamente no convexo \citep{Su2018}.
  \item No todos los diseños son fabricables \citep{Su2020}.
  \item Cada dispositivo es un problema distinto \citep{Molesky2018}.
\end{enumerate}


Así, existe una demanda crítica de un \emph{framework} capaz de optimizar dispositivos con un elevado número de parámetros dentro de un espacio de búsqueda no convexo \citep{Kudyshev2020}. Este es un problema muy grande, por ello en la presente tesis nos centraremos en optimizar un \emph{bend-90°} y un \emph{2-splitter}.

\leonidas{TODO: Agregar una imagen que contenga un bend-90° y un 2-splitter tradicional y uno donde se haya realizado la división en cuadrados para optimizarlos}

\section{Justificación}

El \emph{bend-90°} y \emph{2-splitter} se han escogido debido a que han sido estudiados usando diseño inverso desde el 2004, pero aún no han encontrado aplicación industrial \citep{Molesky2018}. 

Por otro lado, desde el punto de vista computacional este problema es interesante porque ya hay estrategias computacionales conocidas para resolverlo, desde algoritmos evolutivos \citep{Hansen2016} hasta redes neuronales \citep{Goodfellow2015} y \emph{depth learning} \citep{Malkiel2018}. 
Además, debido al alto costo computacional de las simulaciones \citep{Schneider2019}, el trabajo requiere de computación de alto desempeño.
Así, es probable que se pueda obtener buenos resultados en la investigación aplicando el conocimiento ya existente en computación.

\section{Objetivos}

\begin{itemize}

  \item Evaluar y comparar el \emph{performance} y la convergencia de las técnicas modernas de optimización usadas para optimizar un \emph{bend-90°} y \emph{2-splitter}.

  \item Fabricar el diseño con mejor \emph{performance} que se obtenga del \emph{bend-90°} y del \emph{2-splitter} para poder comparar las simulacionales computacionales con las mediciones físicas.

%El siguiente objetivo es implementar un modelo de \emph{deep learning} que pueda predecir el \emph{performance} de un \emph{bend-90°} y de un \emph{2-splitter} con un error cuadrático medio menor a \leonidas{TODO: INVESTIGAR CUANTO DEBERÍA SER}. 

\end{itemize}



\section{Aportes}

Ante la reducida cantidad de trabajos similares en el área \citep{Schneider2019, Elsawy2020}, este trabajo busca aportar una comparativa de las distintas técnicas de optimización que se aplican para optimizar un \emph{bend-90°} y un \emph{2-splitter}. 
