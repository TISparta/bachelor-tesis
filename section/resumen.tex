\chapter*{\center \Large \vspace{-4.5cm} RESUMEN}
\addcontentsline{toc}{section}{\bfseries RESUMEN}
\markboth{RESUMEN}{RESUMEN} 

La fotónica en silicio es un área en desarrollo emergente y constante en las últimas décadas. Los dispositivos fotónicos muestran potencial de aplicación para mejorar
los sistemas de cómputo, telecomunicaciones y otras áreas.
Sin embargo, aún es un reto integrar una gran cantidad de dispositivos fotónicos fundamentales en un chip con área reducida y baja pérdida. 
En el presente trabajo se diseñaron dos dispositivos fundamentales: (i) \emph{bend} y (ii) \emph{2-channel wavelength-demultiplexer} (WDM).
Los diseños se realizaron en un área de $2 \mu m \times 2 \mu m$
siguiendo una estrategia basada en optimización topológica robusta.

Realizamos la evaluación y comparativa de cinco algoritmos de optimización de primer orden: 
(i) \emph{Limited-memory Broyden–Fletcher–Goldfarb–Shanno with boundaries} (L-BFGS-B), 
(ii) \emph{Method of Moving asymptotes} (MMA), 
(iii) \emph{Covariance Matrix Adapatation Evolution Strategy} (G-CMA-ES), (iv) \emph{Gradient Particle Swarm Optimization} (G-PSO) y (v) \emph{Gradient Genetic Algorithm} (G-GA). Los últimos tres algoritmos son variaciones propuestas a sus versiones
más populares (CMA-ES, PSO y GA) donde se incluye el cálculo de
la gradiente para guiar su proceso de optimización.

En nuestros resultados los diseños mejor optimizados presentan: 
(i) transmitancias mayores al $90 \%$ y robustez ante errores de fabricación de dilatación y erosión, 
(ii) porcentaje de gris menor al $2 \%$ y 
(iii) desempeño consistente y con cambios suaves en un rango de longitudes de onda (1500-1600 nm \emph{bend} y 1250-1600 nm WDM) incluso si se eliminan sus regiones no conexas. 

Estos resultados son prometedores para 
(i) la integración de dispositivos  WDM en un área menor al reportado en el estado del arte ( < $2.8 \mu m \times 2.8 \mu m$) y 
(ii) el diseño de \emph{bends} con menores pérdidas que el diseño intuitivo-tradicional de $1 \mu m$ de radio.

\noindent \textbf{Palabras clave:}\\
\noindent Algoritmos de Optimización, Diseño Inverso, Fotónica Integrada, Métodos Numéricos, Optimización Topológica Robusta.
