\chapter*{\center \Large \vspace{-4.5cm} RESUMEN}
\addcontentsline{toc}{section}{\bfseries RESUMEN}
\markboth{RESUMEN}{RESUMEN} 

La fotónica en silicio es un área que se está desarrollando intensamente en los últimas décadas.
Los dispositivos que se están desarrollando muestran un gran potencial para mejorar
los sistemas de cómputo, telecomunicaciones y otras áreas;
sin embargo, sigue siendo un reto incorporar una gran cantidad de estos en un chip con área reducida.
Ante ello, distintas investigaciones buscan obtener dispositivos fotónicos fundamentales 
compactos y con poca pérdida. 
En el presente trabajo se diseñaron dos de estos: (i) \emph{bend} y (ii) \emph{2-channel wavelength-demultiplexer}.
Los diseños se realizaron en un área de $2 \mu m \times 2 \mu m$
siguiendo una estrategia basada en optimización topológica robusta.
Además, se realizó una comparativa de dos algoritmos de optimización de primer orden populares en el área.
Y, se añadió a la comparativa tres variantes de algoritmos populares, pero que no suelen
aprovechar el cómputo de la gradiente para guiar su proceso de optimización.
Los diseños mejor optimizados consiguieron transmitancias mayores al $90 \%$, son resilientes a errores
de fabricación de dilatación y erosión, poseen un porcentaje de gris menor al $2 \%$
y  funcionan bien en un rango de longitudes de onda incluso si se eliminan sus
regiones no conexas.

\noindent \textbf{Palabras clave:}\\
\noindent Optimización Topológica Robusta, Métodos Numéricos, Algoritmos de Optimización, Diseño Inverso
