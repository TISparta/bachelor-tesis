\chapter*{\center \Large \vspace{-4.5cm} RESUMEN}
\addcontentsline{toc}{section}{\bfseries RESUMEN}
\markboth{RESUMEN}{RESUMEN} 

La fotónica en silicio es un área que se está desarrollando intensamente en los últimas décadas.
Los dispositivos que se están desarrollando muestran un gran potencial para mejorar
los sistemas de cómputo, telecomunicaciones y otras áreas.
Sin embargo, sigue siendo un reto incorporar una gran cantidad de dispositivos en un chip con área
reducida.
Ante ello, distintas investigaciones buscan obtener dispositivos fotónicos fundamentales compactos
y con poca pérdida. En el presente trabajo se diseñaron dos de estos dispositivos:
(i) \emph{bend} y (ii) \emph{2-channel wavelength-demultiplexer}.
Los diseños se consiguieron con optimización 
topológica resiliente.
Además, se realizó una comparativa de cinco algoritmos de optimzación populares en el área al optimizar
estos dispositivos.
Los diseños mejor optimizados consiguieron transmitancias mayores al $90 \%$, son resilientes a errores
de fabricación (dilatación y erosión) y funcionan bien en un rango de longitudes de onda.

\noindent \textbf{Palabras clave:}\\
\noindent Optimización Topológica Resiliente, Optimización, Métodos Numéricos
